\section{Marco teórico}

\subsection{Librerias}
\noindent Entre las herramientas utilizadas para el desarrollo de este laboratorio, se encuentran las librerías de:

\begin{itemize}
	\item stdio.h: Para la entrada, salida de datos y manejo de archivos.
    \item stdlib.h: Utilizada para el manejo de la memoria dinámica
	\item string.h: Utilizada para el manejo de concatenaciones de caracteres.
\end{itemize}

\subsection{MIPS (Procesador)}
\noindent Con el nombre de MIPS, se le conoce a la familia de microprocesadores de arquitectura RISC desarrollados por MIPS Technologies, diseñado para optimizar segmentación en unidades de control y facilitar la generación automática de código maquina por parte de los compiladores.

\subsection{Registro}
\noindent Un registro corresponde al hardware que forma parte del procesador y puede contener una determinada cantidad de bits. Estos ofreces un nivel de memoria más rapido y acotado que la memoria principal, son empleados para controlar las instrucciones de ejecución, manejar el direccionamiento de memoria y proporcionar capacidad aritmética.

\subsection{Punteros}
\noindent Los punteros son elementos bastante utilizados durante el desarrollo del código debido a que permiten trabajar directamente a las direcciones de memoria de las estructuras de datos que tenemos. Se utilizan tanto como para lectura del archivo como para poder almacenar cada elemento de la instrucción en su estructura correspondiente.

\subsection{Lectura y escritura de archivos}
\noindent Para poder realizar la lectura de archivos anexos a los ingresados manualmente por terminal en nuestro código se utilizan funciones específicas llamadas por la librería stdio.h las cuales nos facilitan el acceso a éstos.

\subsection{Memoria dinámica}
\noindent Para un óptimo avance en este código se utilizó memoria dinámica durante la inicialización de las estructuras con el fin de que se pueda trabajar con los valores leídos en el archivo de manera más eficiente. Para esto se utilizó la función malloc() para poder asignar el espacio propio de cada instrucción y la función free() para ir liberando la memoria una vez que ésta ya ha sido utilizada.

\subsection{Estructuras}
\noindent Las estructuras son un tipo de dato utilizado para poder agrupar distintos tipos de información y manejar datos que sería muy complicado de realizar de manera más primitiva. En el desarrollo de este código se utilizan para la implementación de los nodos del árbol ya que así se puede acceder de manera inmediata a la información contenida en éstos.
